\documentclass[12pt,letterpaper]{report}
\begin{document}
Do you still know all the words to the songs on your favourite CD from high school? Have you ever heard part of a song and immediately remembered where you were the last time you heard it? These are just some examples that anecdotally show how robust memory for music can be.
%The robust nature of musical memory is especially apparent in patients with Alzheimer's disease, a disease that results in memory deterioration. Even when other memories are lost, long-term familiarity for melody and music lyrics can be present in severe Alzheimer's disease (Cuddy et al., 2012). 
What makes memory for music so special? 
The initial evidence for a distinct memory for music comes from a number of case studies. In 1996, Peretz described patient CN who suffered bilateral temporal lobe damage leading to a severe, music specific agnosia. CN could recognize lyrics from songs, but did not recognize previously familiar melodies. Her normal performance on tests of music perception (melody or tone discrimination) indicated that her agnosia was in fact music-specific and was not a deficit in the processing of melodic information. In contrast, Patient PM (Finke, Esfahani, \& Ploner, 2012), who had been a professional cellist until he contracted encephalitis, had severe semantic and episodic memory deficits but performed like a healthy musician on a music recognition test. PM could not recognize family or friends but he could differentiate between famous and non-famous musical pieces. These case studies indicate that memory for music can be dissociated from other types of memory. In fact, Peretz \& Coltheart (2003) proposed that humans have a 'musical lexicon' that contains representations of all the musical phrases one has ever heard, and that this musical lexicon is separate from the verbal lexicon, where representations of phonological sounds are stored. 

Neuroimaging techniques have allowed researchers to uncover the neural basis for the separate musical lexicon described by Peretz and Coltheart. Using PET, Groussard et al. (2009) showed that the musical lexicon, and musical semantic memory in general, is sustained by a temporo-prefrontal cortical network. This network showed greater activity during a task where participants rated their level of familiarity with a series of melodies than in a task where participants determined whether two unknown melodies were the same or different. Groussard et al. (2009) hypothesized that the right-sided regions within this network are mainly responsible for holding the melodic traces of familiar tunes, whereas the left-sided regions are responsible for the semantic and associative memories involved in recognizing a musical piece as familiar. The left-sided activation occurred in areas common to those classically shown to be involved in verbal semantic memory (Groussard et al., 2009). In 2010, Groussard et al. conducted a follow-up study using fMRI and showed a clear dissociation between the neural patterns elicited by musical and verbal stimuli. These neuroimaging results supported the theory of Baird and Samson (2009) who suggested that musical memory in patients with Alzheimer?s disease is spared because of the intact functioning of the necessary and specific brain regions that are relatively unaffected by the disease. They suggest that explicit musical memory, that relies on the temporal lobes, is affected by Alzheimer's disease, but other types of musical memory such as procedural musical memory rely on frontal areas and therefore are relatively preserved in Alzheimer's. In 2015, Jacobsen, Fritz, Stelzer, and Turner showed that the caudal anterior cingulate and the ventral pre-supplementary motor areas are involved in the processing of both unknown and known music and that these areas, responsible for encoding musical memory, were relatively spared in a sample of patients with Alzheimer's disease. Together, theses results indicate that not only is memory for music separate from other types of memory, but this dissociation can be seen with neuroimaging techniques. 

The current study uses fMRI to define the neural substrates of musical memory. We trained naive subjects on a set of unfamiliar musical stimuli to induce musical familiarity while controlling the amount of exposure to each stimulus. 
Through comparisons of brain activations of the original unfamiliar stimuli to the newly familiar stimuli we gain a clearer understanding of how brain activity differs when listening to unfamiliar and familiar stimuli. 
Such a tightly controlled study will allow us to understand why memory for music is different from other types of memory. 
The stimuli used in this study will allow us to dissociate the role that the words (lyrics) play in musical memory. Participants will listen to stimuli where words are spoken, words are sung without music (a capella), words are sung with music, and where music is heard without words. 
The 


\end{document}