\chapter{Introduction}
What is it like to be a conscious human in our world? 
%This question has interested researchers and philosophers for decades. 
In the second half of the 20th century, philosophers such as Timothy Sprigge and Thomas Nagel were grappling with exactly this question.
Their work describing an individual's conscious experiences came to a point in Nagel's essay "What is it like to be a bat?" (1974). In his essay, Nagel argues that "an organism has conscious mental states if and only if there is something that it is like to \textit{be} that organism". 
As part of his arguments against the reductionist view of the mind-body problem, Nagel states that consciousness cannot be explained without the subjective character of experience and that these experiences are coloured by the organism's past knowledge (p448).

The attempts to further understand this question in the field of philosophy of mind, focused on understanding an individuals' previous experiences, the context in which the individual exists, and other observable characteristics that contribute to how an individual relates to their world (cite). 
However, the advent of neuroimaging techniques has given researchers the ability to empirically approach the question of "what is it like to be...". 
Neuroimaging techniques such as EEG and fMRI allow us to view the brain activity of an individual while they are experiencing their world. 
The world inside an fMRI scanner may not be representative of a natural environment, but the the results provide a unique window into how an individual consciously processes their world. 
\section{Components of consciousness}
There are two key components to understanding an individual's conscious experience. 
First, the decision must be made regarding whether or not a conscious experience is occurring. 
Second, and arguably the more difficult, is to determine the \textit{quality} of that conscious experience.
These components of consciousness can be explored through the collection of empirical neuroimaging data. 

Using EEG and fMRI researchers have been able to calculate the similarity between the brain activity of individuals watching the same movie or listening to audio stimuli (for example audio book excerpts) (cite, cite). 
This analysis technique is called inter-subject synchrony (cite, cite). 
By testing individuals whose consciousness can be corroborated by behavioural evidence, researchers concluded that it is likely the conscious experience of the storyline of the stimulus that drives the similar brain activity across individuals (cite). 
The foundational assumption behind this method is that a high correlation between different individuals' brain activity can be taken as a marker of consciousness.
Therefore, an individual whose brain activity is highly synchronized to a known group of conscious individuals can also be said to be conscious.

Once the presence of consciousness has been established, the second component of conscious experience is a description of the quality of that experience. 
If we assume that the synchronization of the brain activity of individuals is a marker of consciousness, then the fluctuations present in that brain activity can be taken as descriptions of the quality of that experience.
By manipulating the stimuli that participants are exposed to in an experiment, we can dissect how the brain activity fluctuations map onto the conscious experience of characterstics of the presented stimuli.
Eventually, through numerous tested stimuli and experiments in various modalities, researchers will build a 'bank' of stimulus characteristic to brain signature mappings that will allow us to better understand "what it is like to \textit{be}..." from an objective, empirical standpoint. 
\section{Inter-subject synchrony}
Use this section to describe the current literature and go into details about brain areas that are involved.
%Previous experience with the stimuli reduces the strength of the similarity in brain activity between individuals (cite, cite).
\section{Stimulus choice}
In the existing inter-subject synchrony literature the stimuli that have been most commonly used are movies (cite) or story excerpts from audiobooks and storytelling events (cite). 
One type of stimulus that has not been explored in the context of inter-subject synchrony is music.
In the experiments that follow, I used musical stimuli to probe consciousness and inter-subject synchrony. 
Musical stimuli differ from movies and stories in a variety of ways yet I believed music would be an interesting stimulus to explore inter-subject synchrony between individuals for the reasons outlined below.
 
First, music drives behavioural synchrony. 
In general, individuals are readily able to synchronize their body movements (hand clapping, foot tapping) with the beat, or underlying pulse, of a piece of music (cite, cite). 
The ability of an individual to perceive a beat can be detected through EEG and fMRI measures (cite cite cite). 
If multiple individuals are perceiving the same beat in the music, it is likely that their brain activity during this process will be highly synchronized.

Second, musicians convey stories in music both with and without language. Some musicians may rely on the lyrics of a song to tell their story, while others may use the combinations of various instruments and musical elements (cite). 
In both cases however, the musician is trying to convey some meaning to the listener. 
In the inter-subject synchrony literature using movies and stories researchers concluded that it is likely the conscious experience of the storyline of the stimulus that drives the similar brain activity across individuals (cite). 
If that is the case, then it is possible that the story or the meaning conveyed through music may also drive inter-subject synchrony in a group of individuals. 

Lastly, the way individuals remember music is different from the way other types of stimuli are remembered. 
A clear example of this can be seen in older adults with memory disorders; even in the late stages of Alzheimer's disease some older adults will remember music from their youth when other types of memories have been lost (cite). 
This unique characteristic of music provides an opportunity to explore how memory for experiences over long periods of time affect the strength and the quality of inter-subject synchrony. 

\section{Research Questions}
In my first set of experiments I used musical stimuli to probe how adults experience their world. 
By contrasting musical stimuli with and without language information, I was able to understand what aspects of music are important for driving inter-subject synchrony. 
The manipulation of stimulus characteristics also allowed me to start to understand how these characteristics contribute to the \textit{quality} of consciousness.
In these experiments, I used a paradigm that forced individuals to experience the musical stimuli many times.
This paradigm allowed me to probe how an individual's experience with a stimulus affects how the brain processes the music.
These experiments allowed me to answer two distinct questions. First, how do different parts of music (for example, melody and lyrics) interact with each other to drive inter-subject synchrony? Second, how does an individual's familiarity with the stimulus change how our brains process the different parts of a musical stimulus?

In my second set of experiments, I explored how musical stimuli with and without language is processed by older adults and how that processing is different than in younger adults.
As we age, the way we relate to and process the world around us changes. 
Older adults have a large bank of memories and life experiences to use to understand the world around them and therefore it is different to be an older adult than it is to be a younger adult. 
Previous inter-subject synchrony work with movie stimuli has shown that individuals' brain activity becomes more idiosyncratic with age (Campbell et al).
By using the same set of stimuli as used in the younger adult study, I was able to understand how age changes the way our brains process stimuli as well as how the \textit{quality} of the experience of an older adult differs from that of a younger adult.
Capitalizing on the older adults' long lifespan and increased experience with certain stimuli, I was also able to contrast stimuli participant's had known for decades with unknown stimuli to better understand how long-term experience affects how the brain processes a stimulus.

\section{Summary}
Use this section for a quick summary of the research questions being asked.

QUESTION 1: how do the different parts of music (melody and lyrics) interact with each other to drive synchrony?

QUESTION 2: how does familiarity with a stimulus interact with how our brains process the different parts of musical stimuli?

QUESTION 3: how does age change how our brains process stimuli and what does that tell us about how older adults might be experiencing these stimuli?

QUESTION 4: How does long-known music compare to unknown in older adults and newly learned music in young adults?
%***

%Study 1\&2: SYNCHRONY: Young Adults - stimuli across 4 types (a,i,s,w) with a familiarity manipulation. 
%This study tells us about 1) how the brain synchronizes across the different stimuli types. That spoken words are most important and that music doesn't drive neural synchrony despite driving behavioural synchrony. 2) how familiarity plays a role in synchrony - being more familiar with a stimulus does not strengthen synchrony. 

%Study 2: RSA: Young Adults - learn from synchrony that the brain is responding to these stimuli differently. Ran a searchlight to show which brain areas differentiate between the 4 types of stimuli. Ran a RSA to look at how different the four stimuli types are. 
%Does RSA show anything different for familiar vs unfamiliar? 

%From first two studies we learn that familiarity doesn't play a role in the strength of the brain's responses to the stimuli. 

%Study 3: SYNCHRONY: Older Adults - stimuli across 3 types (i,s,w). Look at how older adult brains synchronize across different stim types and compare findings to young adults. How does this affect how we think about consciousness?

%Study 4: How do the long known songs compare to the unknown songs? 