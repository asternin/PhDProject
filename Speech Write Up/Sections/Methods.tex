\section*{Methods}
\subsection*{Participants}
\subsection*{Stimuli}
Eight stimuli were used during this experiment: four rhythmic patterns and four rhythm-matched speech patterns (See Figure  ). 
In a block design, rhythmic patterns were played during the first half of the experiment and the speech patterns during the second half. 
Each stimulus was preceded by four metronome clicks.
The metronome continued throughout the stimulus, which played for 12 seconds. 
After 12 seconds, the stimulus stopped playing, the metronome continued, and participants were asked to imagine the stimulus just as they had heard it for another 12 seconds. 
Each stimulus was heard 12 times throughout the experiment. 
\subsection*{Equipment and Procedure}
We collected information about participants' previous musical experience, their ability to imagine sounds, and musical sophistication using an adapted version of the Goldsmith's Musical Sophistication Index [] combined with a clarity of auditory imagination scale. 
Participants were seated in an audiometric room (Eckel model CL-13) and the data were collected using a BioSemi Active-Two system with 64+2 EEG channels. 
Horizontal and vertical EOG channels were used to record eye movements. EEG was sampled at 512 Hz. A Cedrus StimTracker was used to ensure minimal delay (<0.05 ms) between the presentation of the stimulus to the participant and the marking of stimulus onset in the data. 
The experiment was programmed and presented using PsychToolbox run in MATLAB 2014a. A computer monitor displayed the instructions and fixation cross and speakers played the stimuli at a comfortable volume for each participant. 
The volume was kept constant across participants.
\subsection*{EEG Data Processing}
EEG pre-processing was done using EEGLab. 
Data were filtered between 0.5Hz and 30Hz, downsampled to 256Hz, epoched to remove break-periods, and submitted to an ICA analysis.
ICA components corresponding to artefacts (eyeblinks, heart rate, etc.) were manually removed.
Clean EEG data was exported from EEGLab into Fieldtrip for frequency analyses.