\section*{Introduction}
The search for a method to convey our thoughts without employing speech or actions has preoccupied researchers for decades.
In particular, patients who are fully conscious and awake, yet are unable to respond behaviourally due to brain damage \cite{owen_2006,monti_2010,cruse_2011}, expose the necessity for alternate forms of communication. 
Over the past 30 years, \ac{EEG} studies have provided significant insights into how ``neural responsivity'' might be used to drive motor-independent communication.
\acp{BCI} using \ac{EEG} rely on large scale changes in brain states like P300 evoked potentials \cite{farwell_1988,mugler_2010,tanaka_2005}  and sensorimotor rhythms \cite{blankertz_2010, pfurtscheller_2001} for gross control of an external device.
Other \acp{BCI} use signals such as visual evoked potentials \cite{miranda_2011, wang_2006} allowing users to make a binary choice. 

A binary choice system has its limitations however, and creating a \ac{BCI} that allows patients to communicate via more than two options is needed.
This requires inducing more than two distinct brain states.
Results from previous studies show that perceived rhythmic tonal patterns create distinct patterns in an EEG signal that can be identified \cite{stober2014}.
Such distinct patterns could be used to control a \ac{BCI} but the rhythms must be actively induced by the patient. 
Therefore, imagination must be involved.
Imagining a tonal rhythm may be difficult for a patient to do, so a stimulus that is more salient and easier to imagine is necessary.
Speech is inherently rhythmic and imagination of rhythmic speech phrases could be used to induce distinct patterns in the brain.
There is some precedent as both heard \cite{Formisano2008} and imagined \cite{Deng2010} rhythmic patterns of vowel sounds have been classified from EEG signals. 

In this experiment, participants listened to and imagined short, repeated rhythms composed of tones and short, repeated speech phrases to induce different brain states.
We investigated whether we could decode a user's \ac{EEG} signal to determine which rhythm or speech phrase they were listening to or imagining.

Information on phase here

Using information from the phase domain of the EEG signal we...